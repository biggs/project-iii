%Michelmas report.tex
\documentclass[12pt,twoside,a4paper,english]{article}
\usepackage[latin9]{inputenc}
\usepackage{amsbsy}
\usepackage{amssymb}
\usepackage{babel}

\begin{document}

\title{Michelmas Initial Report: Exploring the Dynamical Behaviour of Spin Waves Through Energy Absorbtion Interferometry}
\author{Felix Biggs - fb369 (EM) \\ Supervisior: S. Withington}
\date{December 2014}
\maketitle

\abstract

Energy absorbtion interferometry is a general technique, based on the physics of partially coherent energy absorbtion, that can be used to find the natural dynamical modes of an self-interacting system.
It has previously been used to find the modes of interacting electric dipoles [S. Withington and C. N. Thomas, Probing the dynamical behaviour of surface dipoles through energy-absorption interferometry, Phys Rev A, 86, 043835, 2012].
It has been shown theoretically that the same technique can be applied to systems of interacting magnetic dipoles, and therefore can be used to explore the complex and fascinating world of spin waves and spintronic devices.
The aim of this project is to simulate and investigate such systems.


\section{Introduction}

\subsection{Spin Waves}

Spin waves are coherent excitations of magnetic dipoles, with scale lengths from $O(10^{-4})$m to $O(10^{-8})$m and characteristic frequencies from gigahertz to terahertz. At low frequencies they are mediated by dipole-dipole interactions and at high frequencies exchange interactions.
They can be created by appying a large static field and a small time varying field to a magnetic material - the static field causes the dipoles to align, or at least form domains, and the time-varying field causes the dipoles to precess in the form of a disturbance that propogates as a waves.
They were originally studied in bulk materials but recntly focus has shifted towards engineering microscopic band gaps to perform operations such as data manipulation and storage. The unusual properties of single and multiple nanoparticles are of particular interest.

\subsection{Energy Absorbtion Interferometry}
The general idea is that the magnetic material is placed in a large static magnetic field, and a pair of phase-locked, time-varying sources - say magnetic dipoles - are placed nearby.
The total power absorbed is measured as a function of the phase difference between the sources, giving a fringe. The complex visibility of this fringe is measured for pairs of source locations, using which the technique is able to reconstruct the dynamical modes and their relative responsivities.
Power absorbed can be measured very accurately for a thin film by placing it on a thermally isolated Si or SiN$_x$ membrane and using bolometer technology.

The most interesting thing about the technique is that only the total absorbed power needs to be measured, making this applicable to a large number of systems, for example electric dipoles - as has already been demonstrated [S. Withington and C. N. Thomas, Probing the dynamical behaviour of surface dipoles through energy-absorption interferometry, Phys Rev A, 86, 043835, 2012].


\section{Theory and Application}

We will use a discretised model or a system of magnetic dipoles (which may be elements of a discrete dipole approximation). The column vector \(h^e \in \mathbb{C}^{2J} \) contians the complex amplitudes of the cartesian field components of the external field at the position of the dipoles.

\subsection{Defining System}

\begin{enumerate}

\item \textbf{Non-interacting case}
For a collection of non-interacting dipoles, it is found that:

\[
\overline{\overline{\chi}}(\boldsymbol{r})=\sum_{j}\frac{\gamma_{j}\mu_{0}m_{j}^{s}\omega}{(\omega_{0j}^{2}-\omega^{2})-i\omega\Gamma_{j}}\overline{\overline{S}}_{j}\delta(\boldsymbol{r-\boldsymbol{r_{0j}}})
\]

and 

\[
\overline{\overline{S}}_{j}=\frac{\omega_{0j}}{\omega}\boldsymbol{\hat{i}\hat{i}}-i\boldsymbol{\hat{j}\hat{i}}+i\boldsymbol{\hat{i}\hat{j}}+\frac{\omega_{0j}}{\omega}\boldsymbol{\hat{j}\hat{j}}
\]

is the spin dyadic.
When we expand in the basis of \(\textstyle{h}^{e}\), $\chi$ becomes a block diagonal matrix. This
represents the system in the non-interacting case.


\item \textbf{Self interacting case}
The exchange interactions are added using another matrix, ${\textstyle G^{x}}$-
in the case of nearest neighbour interactions, simply the sum over
local dipole moments (represented in the vector $m\in\mathbb{C}^{2J}$
). Thus:

\[
{\textstyle m}=\chi{\textstyle h^{e}}+\chi{\textstyle G^{x}m}=(\chi^{-1}-{\textstyle G^{x}})^{-1}{\textstyle h^{e}}=\xi{\textstyle h^{e}}
\]

This defines \(\xi\).



\item \textbf{Case with Scattering}

The scattered magnetic dipole field is calculated using the discretised
Green's function, ${\textstyle G^{d}}$ using the magnetostatic Green's
dyadic:

\[
\overline{\overline{\boldsymbol{G}}}(\boldsymbol{r},\boldsymbol{r'})=\frac{1}{|\boldsymbol{r}-\boldsymbol{r'}|^{3}}[3\overline{\overline{\boldsymbol{R}}}(\boldsymbol{r},\boldsymbol{r'})-\overline{\overline{\boldsymbol{I}}}]
\]

the total field at the dipoles is thus given by 

\[
h^{t}=h^{e}+G^{d}m=(I-G^{d}\xi)^{-1}h^{e}=\kappa h^{e}
\]

which defines $\kappa$, and thus also $m=\xi\kappa h^{e}$.



\item \textbf{Response Matricies of systems}

We now define the three response matricies for the increasingly complex
systems:

\[
N=\frac{\omega}{2}\mu_{0}\chi^{R}
\]

\[
M=\frac{\omega}{2}\mu_{0}\xi^{R}
\]

\[
L=\frac{\omega}{2}\mu_{0}\kappa^{\dag}\xi^{R}\kappa
\]

where $\chi^{R},\xi^{R}$ are the hermitian parts of $-i\chi,-i\xi$
respectively. These give the absorbed power for the systems:

\[
<P>=\textrm{Tr}\left[H^{e}N\right]
\]

\[
<P>=\textrm{Tr}\left[H^{e}M\right]
\]

\[
<P>=\textrm{Tr}\left[H^{e}L\right]
\]

where $H^{e}$ is the ensemble average over the outer product $<h^{e}h^{e*}>$.
\end{enumerate}

Thus the system is defined by the individual dipole properties, their exchange interaction strengths and their locations.

\subsection{Response and reconstruction}
Use the magnetostatic Green's function to calculate the magnetic field at each dipole for a given source configuration (only use either one or two point sources here). This is stored in the \(H^e\) matrix.
A reconstruction of the modes can be found using
\[L' = \widetilde{H}^e A \widetilde{H}^{e\dagger}  \]
where \(A\) is the response matrix, with elements for a given source configuration.
As \(H^e\) may not be square we must use SVD (single value decomposition) to calculate \(\widetilde{H}^e\), which reflects that a certain number of response coefficients is required to reconstruct the modes.


\section{Project Plan and Timeline}
\begin{enumerate}
\item Apply the method with specific code to some simple systems, e.g. a pair of dipoles.
\item Create generalised code capable of simulating a system of dipoles and their interactions/responses
\item Create code capable of running EAI on such a system
\item Compare results with theoretical for a number of possible dipole systems, with an emphasis on wavelike (linear) systems
\item Create plots of magnetic field for various modes about such systems
\item Investigate effects of scattering in reducing recoverable information
\end{enumerate}
Aim to complete items 1 to 3 early in Lent to allow a number of different systems to be explored. No special equipment other than MATLAB required.

\end{document}