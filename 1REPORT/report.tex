%% LyX 2.1.2.2 created this file.  For more info, see http://www.lyx.org/.
%% Do not edit unless you really know what you are doing.
\documentclass[english]{article}
\usepackage{amsmath}
\usepackage{amssymb}
\usepackage{wasysym}
\usepackage{esint}
\usepackage{babel}
\begin{document}

\title{Exploring the Dynamical Behaviour of Spin Waves Through Energy Absorption
Interferometry}


\author{Felix Biggs}
\maketitle
\begin{abstract}
The layout suggested is: 1. Introduction 2. Review 3. Methodology
4. Theory 5. Results sections 6. Discussion 7. Conclusion 8. References
9. Appendices 
\end{abstract}

\section{Introduction}


\subsection{Spin Waves}

Spin waves are coherent excitations of magnetic dipoles, with scale
lengths from $O(10^{-4})$m to $O(10^{-8})$m and characteristic frequencies
from gigahertz to terahertz.

At low frequencies they are mediated by dipole-dipole interactions
and at high frequencies exchange interactions. They can be created
by applying a large static field and a small time varying field to
a magnetic material - the static field causes the dipoles to align,
or at least form domains, and the time-varying field causes the dipoles
to precess in the form of a disturbance that propagates as a waves.
They were originally studied in bulk materials but recently focus
has shifted towards engineering microscopic band gaps to perform operations
such as data manipulation and storage. The unusual properties of single
and multiple nanoparticles are of particular interest.


\subsection{Ferromagnetic Resonance}

The current primary technique for investigation of spin wave systems
is ferromagnetic resonance. In this technique -.... We model such
a etc


\subsection{Energy Absorption Interferometry}

The general idea is that the magnetic material is placed in a large
static magnetic field, and a pair of phase-locked, time-varying sources
- say magnetic dipoles - are placed nearby. The total power absorbed
is measured as a function of the phase difference between the sources,
giving a fringe. The complex visibility of this fringe is measured
for pairs of source locations, using which the technique is able to
reconstruct the dynamical modes and their relative responsivities.
Power absorbed can be measured very accurately for a thin film by
placing it on a thermally isolated Si or SiN\_x membrane and using
bolometer technology.

The most interesting thing about the technique is that only the total
absorbed power needs to be measured, making this applicable to a large
number of systems, for example electric dipoles - as has already been
demonstrated {[}S. Withington and C. N. Thomas, Probing the dynamical
behavior of surface dipoles through energy-absorption interferometry,
Phys Rev A, 86, 043835, 2012{]}.


\subsection{In Practice}

A practical system for performing EAI might look like: ????

Bolometer technology is used to measure the power absorped through
very accurate temperature sensing


\section{Theory}


\subsubsection*{Short Note on Notation}

Vectors in bold font, with hats for unit vectors: ${\bf u},{\bf m}_{j},\hat{{\bf k}}$

Scalars are (slightly) italicised: $H,\omega$

Dyadics (outer products of vectors): bold with a double overbar: $\overline{\overline{\boldsymbol{\chi}}},\overline{\overline{\boldsymbol{{\bf H}}}}$

Matrix representations of dyadics and scalars: normal font: $\mathrm{h},\mathrm{H^{tot}},\mathrm{\chi}$


\subsection{Spin Waves}

We shall model the system as a discrete system of magnetic dipoles.
This may be done for any system using the discrete dipole approximation
(DDA) - the equation then represents averages over finite macroscopic
regions (check/write more????). A single dipole with magnetic dipole
moment $\mathrm{{\bf m_{j}}}$ will evolve according the Landau-Lifshitz-Gilbert
equation:
\begin{equation}
\frac{d{\bf {m_{j}}}}{dt}=-\gamma_{j}\mu_{0}{\bf {m_{j}}}\times{\bf {H}}_{j}^{tot}+\frac{\alpha_{j}}{|{\bf {m_{j}}}|}({\bf {m_{j}}}\times\frac{{d{\bf {m_{j}}}}}{dt})
\end{equation}
$j$ is an index which specifies a dipole, ${\bf H}_{j}^{tot}$ is
the total magnetic field at the dipole, $\gamma_{j}$ (positive and
negative values corresponding to opposite rotation directions) is
the gyromagnetic ratio of the dipole and $\alpha_{j}$ is a damping
coefficient (in FMR, for the transverse components of the dynamic
part of the magnetization). This equation could be generalized to
a Bloch equation with different longitudinal and transverse dampings
but this will not be done to simplify the analysis.

We move to the steady state by writing ${\bf H}_{j}^{tot}={\bf H}_{j}^{tot\,(0)}+{\bf H}_{j}^{tot\,(t)}$
which represent the non-time dependent and time dependent terms respectively,
provided by the large static field and the smaller RF sources. Thus
when only the static field is present, the system equilibrates into
a state $\mathrm{{\bf m_{j}}}^{(0)}$ with no torque on dipoles (from
LLG). Generally, the dipoles can form non-aligned domains but we assume
they are in the saturated state: all totally aligned with the static
field, so $\mathrm{{\bf m_{j}}}^{(0)}=m^{s}\hat{{\bf k}}$ , and ${\bf H}^{tot(0)}\equiv H^{0}\hat{{\bf k}}\approx{\bf H}^{ext(0)}$.
We also make the vital assumption that in the steady state, all of
the absorbed (or indeed extracted) power is transferred from substrate
(physically, its phonon system).

We use periodic RF sources so that ${\bf H}_{j}^{tot\,(t)}={\bf H}_{j}^{tot\,(1)}e^{-i\omega t}$.
Substituting into LLG and discarding non-linear terms, effectively
the approximation that the dynamical component $\mathrm{{\bf m_{j}}}$
is much smaller than the static component in the same direction (we
keep only the transverse dynamical components):

\[
i\frac{\omega}{\gamma\mu_{0}}{\bf m}^{(1)}=m^{s}\hat{{\bf k}}\times{\bf H}^{tot\,(1)}+{\bf m}^{(1)}\times H^{0}\hat{{\bf k}}+i\frac{\omega\alpha}{\gamma\mu_{0}}\hat{{\bf k}}\times{\bf m}^{(1)}
\]



\subsubsection*{Non-Interacting Case}

We may invert the above equation to write the susceptibility of a
system of non-interacting, precessing dipoles with position vectors
${\bf r}_{0j}$ as:

\begin{equation}
\overline{\overline{\boldsymbol{\chi}}}(\boldsymbol{r})=\sum_{j}\frac{\gamma_{j}\mu_{0}m_{j}^{s}\omega}{(\omega_{0j}^{2}-\omega^{2})-i\omega\Gamma_{j}}\overline{\overline{\textrm{\ensuremath{\boldsymbol{S}}}}}_{j}\delta(\boldsymbol{r-\boldsymbol{r_{0j}}})\label{eq:localx}
\end{equation}
 with 
\[
\overline{\overline{\boldsymbol{\textrm{S}}}}_{j}=\frac{\omega_{0j}}{\omega}\boldsymbol{\hat{i}\hat{i}}+i\boldsymbol{\hat{j}\hat{i}}-i\boldsymbol{\hat{i}\hat{j}}+\frac{\omega_{0j}}{\omega}\boldsymbol{\hat{j}\hat{j}},\quad S_{j}=\left(\begin{array}{cc}
\omega_{0}/\omega & -i\\
i & \omega_{0}/\omega
\end{array}\right)
\]
 so that the spatial dipole moment is ${\bf M}^{(1)}({\bf r})=\overline{\overline{\chi}}({\bf r})\cdot{\bf H}^{(1)}({\bf r})$.
Note here there is no spatial integral or ${\bf r'}$ dependence as
the non-interacting case is totally local. $S_{j}$ is the matrix
form in the usual representation. The ${\bf m}_{j}^{(1)}$ are time
independent in the plane perpendicular to ${\bf H}^{(0)}$. The natural
frequency of precession is $\omega_{0j}=\gamma_{j}\mu_{0}H_{j}^{(0)}$
and the damping rate is $\Gamma_{j}=2\alpha_{j}\omega_{0j}$, and
we have assumed low loss (small $\alpha_{j}$).

The natural modes of the isolated dipole are found from the eigenvectors
and eigenvalues (diagonalisation) of $\overline{\overline{{\bf S}}}$:

\[
{\bf u}_{+}=\frac{1}{\sqrt{2}}(\hat{{\bf i}}+i\hat{{\bf j}}),\quad\lambda_{+}=\frac{\omega_{0}}{\omega}+1
\]


\[
{\bf u}_{-}=\frac{1}{\sqrt{2}}(\hat{{\bf i}}-i\hat{{\bf j}}),\quad\lambda_{-}=\frac{\omega_{0}}{\omega}-1
\]
 corresponding to rotation in the the natural precession direction,
and in the reverse direction respectively.


\subsubsection*{Interacting Case}

We can write ${\bf H}_{j}^{tot}={\bf H}_{j}^{ext}+{\bf H}_{j}^{xch}+{\bf H}_{j}^{dip}$
where the terms represent the applied field to the system, the effective
field from exchange interaction with the other dipoles, and the scattered
field from the other dipoles.

The exchange interaction effective field can be written as:
\begin{equation}
{\bf H}_{j}^{xch}=J\sum_{k=NN}{\bf m}_{k}\label{eq:exchange}
\end{equation}
 where NN indicates the nearest neighbours.

The dipole scattering term is:
\[
{\bf H}^{dip(1)}({\bf r)}=\int\overline{\overline{{\bf G}}}({\bf r},{\bf r'})\cdot{\bf M}^{(1)}({\bf r'})d^{3}{\bf r'}
\]
 using the non-retarded magnetostatic greens dyadic
\begin{equation}
\overline{\overline{{\bf G}}}({\bf r},{\bf r'})=\frac{1}{|{\bf r}-{\bf r'}|^{3}}\left(3\overline{\overline{{\bf R}}}({\bf r},{\bf r'})-\overline{\overline{{\bf I}}}\right),\qquad\overline{\overline{{\bf R}}}({\bf r},{\bf r'})=????\label{eq:magnetostatic}
\end{equation}
 which may be derived from (reference, also what is R, and is it correct
in my code????).

We may now combine these terms and the non interacting succeptebility
to give:
\begin{equation}
{\bf H}^{ext(1)}({\bf r}_{i})=\sum_{j}\overline{\overline{{\bf T}}}_{ji}\cdot{\bf H}^{tot(1)}({\bf r}_{j})\label{eq:tdef}
\end{equation}
\[
\overline{\overline{{\bf T}}}_{ij}=\overline{\overline{{\bf I}}}\delta_{ij}-\frac{\gamma_{i}\mu_{0}m_{i}^{s}\omega}{(\omega_{0i}^{2}-\omega^{2})-i\omega\Gamma_{i}}\left\{ J\delta_{i,NN}\overline{\overline{{\bf S}}}_{i}+(1-\delta_{ij})\overline{\overline{{\bf G}}}({\bf r}_{j},{\bf r}_{i})\cdot\overline{\overline{{\bf S}}}_{i}\right\} 
\]
which can be written as a matrix equation in the dipole locations,
and inverted to give the total field at each dipole in terms of the
applied field.


\subsubsection*{Power Absorption}

Since EAI uses power absorption measurements, calculation of them
is needed. In the explicitly time dependent and real case, the instantaneous
power absorption is:
\[
P(t)=\mu_{0}\int{\bf H}({\bf r},t)\cdot\frac{\partial{\bf M}({\bf r},t)}{\partial t}d^{3}{\bf r}
\]
 and thus the time average power absorption is 
\begin{equation}
P(\nu)=\frac{\omega}{2}\mu_{0}\mathrm{Im}\varint_{V}{\bf H}^{*}({\bf r})\cdot{\bf M}({\bf r})d^{3}{\bf r}=\frac{\omega}{2}\mu_{0}\mathrm{Im}\varint_{V}\varint_{V}{\bf H}^{*}({\bf r})\cdot\overline{\overline{\boldsymbol{\chi}}}(\boldsymbol{r},{\bf r'})\cdot{\bf H}({\bf r'})d^{3}{\bf r}d^{3}{\bf r'}
\end{equation}
 which is over the volume of the sample, $V$, since the succeptibility
elsewhere is zero.

For our system of dipoles we thus find a power absorption:
\[
P(\nu)=\sum_{j}\frac{\gamma_{j}\mu_{0}^{2}m_{j}^{s}}{2}\left[\frac{\omega\Gamma_{j}}{\omega^{2}(\frac{\omega_{0j}^{2}}{\omega^{2}}-1)^{2}+\Gamma_{j}^{2}}\right]\times{\bf H}_{j}^{tot(1)*}\cdot\overline{\overline{{\bf S}_{j}}}\cdot{\bf H}_{j}^{tot(1)}
\]
 If the system is rotationally excited, ${\bf H}^{tot(1)}({\bf r}_{i})=a{\bf u}_{\pm}$,
the absorbed power will be
\[
<P(a{\bf u}_{\pm})>=\frac{1}{2}a^{2}\Gamma\gamma\mu_{0}^{2}m_{s}\frac{\omega(\omega_{0}/\omega\pm1)}{\omega^{2}(\omega_{0}^{2}/\omega^{2}-1)^{2}+\Gamma^{2}}
\]
 We see from the numerator that the power is always positive for rotation
with the natural precession direction, and reaches a peak at $\omega=\omega_{0}$.
However, for the opposite direction rotations, the power is zero at
$\omega=\omega_{0}$, and work may be done by the system for $\omega>\omega_{0}$,
taking energy out of the dipole. Physically this corresponds to ????


\subsection{Theory of EAI}

After a number of steps from the above forms for the power we may
arrive at the most useful form: 
\begin{equation}
<P(\nu)>=\frac{\omega}{2}\mu_{0}\varint_{V}\varint_{V}\overline{\overline{{\bf C}}}^{tot(1)}(\boldsymbol{r},{\bf r'})\,\cdot\,\cdot\,\overline{\overline{\boldsymbol{\chi}}}^{R}(\boldsymbol{r},{\bf r'})d^{3}{\bf r}d^{3}{\bf r'}\label{eq:dyadicproject}
\end{equation}
 where we have defined $\overline{\overline{\boldsymbol{\chi}}}^{R}(\boldsymbol{r},{\bf r'})=-i\,\overline{\overline{\boldsymbol{\chi}}}^{A}(\boldsymbol{r},{\bf r'})$
with $\overline{\overline{\boldsymbol{\chi}}}^{A}$ the antihermitian
part of $\overline{\overline{\boldsymbol{\chi}}}$ (this makes the
power real and discards the non-absorping hermitan part of $\overline{\overline{\boldsymbol{\chi}}}$);
and the magnetic field correlation dyadic (the expectation value across
an ensemble of systems, or no such step required for coherent sources):
\begin{equation}
\overline{\overline{{\bf C}}}(\boldsymbol{r},{\bf r'})=<{\bf H}({\bf r}){\bf H}^{*}({\bf r'})>
\end{equation}
 and the double dot is a full contraction to a scalar.

This clearly has the form of an inner product in the abstract vector
space of dyadic fields. The power absorbed is the projection of the
magnetic field correlation dyadic onto the succeptibility dyadic.


\subsubsection*{Non-Scattered Case}

If a source at ${\bf r}_{n}$ produces the field ${\bf h}_{n}({\bf r})$,
and we have $N$ possible source positions, we produce a basis $\mathbb{A}=\{{\bf h}_{n}({\bf r}),\forall n\in1...N\}$.
We can then write the matrix elements of $\overline{\overline{\boldsymbol{\chi}}}^{R}(\boldsymbol{r},{\bf r'})$
in the basis, and approximately (or exactly, if the source fields
span the basis of absorption) reconstruct it using the dual vector
set:

\[
\chi_{nm}=\varint_{V}\varint_{V}{\bf h}_{n}^{*}({\bf r})\cdot\overline{\overline{\boldsymbol{\chi}}}^{R}(\boldsymbol{r},{\bf r'})\cdot{\bf h}_{m}({\bf r'})d^{3}{\bf r}d^{3}{\bf r'}
\]
\begin{equation}
\overline{\overline{\boldsymbol{\chi}}}^{R}(\boldsymbol{r},{\bf r'})\approx\sum_{nm}\chi_{nm}{\bf \tilde{h}}_{m}({\bf r}){\bf \tilde{h}}_{m}^{*}({\bf r'})
\end{equation}


Since we can find the dual set $\mathbb{\tilde{A}}$ numerically from
knowing the forms of the impressed fields, if we know the matrix elements
$\chi_{nm}$ we can find the response dyadic and diagonalise it to
find the natural modes of the system.

It turns out we can do this using only power measurements: this is
the beauty of EAI. We illuminate the sample with two fully coherent
sources so that (???? tot or ext) 
\[
{\bf H}^{ext????}({\bf r}_{j})={\bf h}_{n}+{\bf h}_{m}e^{-i\Delta\phi}
\]
 where $\Delta\phi$ is a phase that we may rotate. We then find that
the absorbed power becomes:
\[
<P(\nu)>=\frac{k_{0}Z_{0}}{2}\left\{ \chi_{nn}+\chi_{mm}+2|\chi_{nm}|\cos(\Delta\phi+\arg(\chi_{nm}))\right\} 
\]


Thus $\chi_{nm}$ is a hermitan matrix, and we may measure the on
diagonal terms directly from the power absorbance from each source,
and the off diagonal terms by sweeping the phase between the sources
and measuring the resulting fringe in the power.


\subsubsection*{Scattering}

In the scattered case????


\subsection{Matrix Formulation}

Now that our model is fully discretised, we describe it in terms of
matricies. The column vectors $\mathrm{h^{ext}},\mathrm{h^{tot}}\in\mathbb{C}^{2J}$
contain the complex amplitudes of the transverse (to the static field,
now placed so ${\bf \hat{k}}={\bf \hat{z}}$, also where is time dependence,
have to take real part????) cartesian field components of the external
and total fields at the position of the dipoles. Clearly from \ref{eq:tdef}
we can write $\mathrm{h^{tot}}=\mathrm{T}^{-1}\mathrm{h^{ext}}$,
and also: 
\[
P=\frac{\omega}{2}\mu_{0}(\mathrm{h^{tot}})^{\dagger}\chi^{R}\mathrm{h^{tot}}
\]
\[
<P>=\mathrm{Tr}\left[\mathrm{C^{tot}N}\right],\qquad\mathrm{N}=\frac{\omega}{2}\mu_{0}\chi^{R}
\]
 where we have taken then trace of a number and used the cyclic property
to write $\mathrm{C^{tot}}=<\mathrm{h^{tot}}(\mathrm{h^{tot}})^{\dagger}>$.
Conceptually, this is equivalent to equation \ref{eq:dyadicproject}.

We may also write the power in terms of the external field:
\[
P=\frac{\omega}{2}\mu_{0}(\mathrm{h^{tot}})^{\dagger}\chi^{R}\mathrm{h^{tot}}=\frac{\omega}{2}\mu_{0}(\mathrm{h^{ext}})^{\dagger}(\mathrm{T}^{-1})^{\dagger}\chi^{R}(\mathrm{T}^{-1})\mathrm{h^{ext}}=(\mathrm{h^{ext}})^{\dagger}\mathrm{L}\mathrm{h^{ext}}
\]
\begin{equation}
<P>=\mathrm{Tr}\left[\mathrm{C^{ext}L}\right]
\end{equation}


Since $\mathrm{C^{ext}}$ and $\mathrm{L}$ are hermitian, they may
be diagonalised: $\mathrm{C^{ext}}=\sum_{i}\alpha_{i}\mathrm{f}_{i}\mathrm{f}_{i}^{\dagger}$,
$\mathrm{L}=\sum_{i}\beta_{i}\mathrm{g}_{i}\mathrm{g}_{i}^{\dagger}$.
The power is then
\[
<P>=\sum_{ij}\alpha_{i}\beta_{j}\left|\mathrm{f}_{i}^{\dagger}\mathrm{g}_{j}\right|^{2}
\]
 which describes how the impressed field modes project onto the system
modes.


\subsubsection*{Closer examination of L}

We may look more closely at exactly where $\mathrm{L}$ comes from
and how it may be incrementally built up. We have a column vector
$\mathrm{m}\in\mathbb{C}^{2J}$ containing the transverse components
of the dipole moments. If exchange interactions are present then:
\[
\mathrm{m}=\chi\mathrm{h^{ext}}+\chi\mathrm{G^{xch}m}=(\chi^{-1}-\mathrm{G^{xch}})^{-1}\equiv\xi\mathrm{h^{ext}}
\]
 where $\mathrm{G^{xch}}$ is the exchange interaction from \ref{eq:exchange}
and $\chi$ is simply the local succeptibility. 

Next we include dipole-dipole scattering, with \ref{eq:magnetostatic}
giving us $\mathrm{G^{dip}}$:
\[
\mathrm{h^{tot}}=\mathrm{h^{ext}}+\mathrm{G^{dip}m}=(1-\mathrm{G^{dip}}\xi)^{-1}\equiv\kappa\mathrm{h^{ext}}
\]
 defining $\kappa$, and giving $m=\xi\kappa h^{e}$.

Thus

\begin{equation}
\mathrm{L}=\frac{\omega}{2}\mu_{0}\kappa^{\dag}\xi^{R}\kappa
\end{equation}
where $\xi^{R}$ is the hermitian part of $-i\xi$.

Finally, for convinience of computation, we also introduce the matrix:

\[
\mathrm{M}=\frac{\omega}{2}\mu_{0}\xi^{R}
\]



\subsection{Reconstruction with EAI}

If the matrix $\mathrm{H^{ext}}\in\mathbb{C}^{2J\times2N}$ (????
naming?) is made up of columns containing the fields $\mathrm{h^{ext}}$
associated with particular possible sources. If these impressed fields
span the absorption modes, the response matrix can be written in terms
of the duals:
\begin{equation}
\mathrm{L}=\sum_{nm}a_{nm}\mathrm{\tilde{h}_{n}^{ext}}\mathrm{\tilde{h}_{m}^{ext\dagger}}=\mathrm{\tilde{H}A\tilde{H}^{\dagger}}
\end{equation}
 so $\mathrm{A}$ gives the response in the basis of the sources.

The power absorbed due to two sources is given by:

\[
P=[\mathrm{h_{1}^{ext\dagger}}e^{-i\phi_{1}}+\mathrm{h_{2}^{ext}}e^{-i\phi_{2}}]\,\mathrm{L}\,\mathrm{[h_{1}^{ext}}e^{i\phi_{1}}+\mathrm{h_{2}^{ext\dagger}}e^{i\phi_{2}}]
\]
 and subbing in the response:

\[
P_{nm}=a_{nn}+a_{mm}+a_{nm}e^{i\Delta\phi}+a_{mn}e^{-i\Delta\phi}
\]
 so we may measure a fringe in the power by rotating the phase, and
hence find the matrix elements of $\mathrm{A}$.


\subsubsection*{Deconvolution}

The matrix $\mathrm{A}$ contains the intrinsic absorbances of the
dipoles, $\mathrm{N}$, convolved with the scattering and exchange
interactions, $\mathrm{T}^{-1}$, and finally the greens function
taking the field at the sources to the dipoles. Generally only the
final step may be deconvolved, as if we know the source properties
we may calculate the forms of the impressed fields and invert them
(we are allowed to do this because in general we do not only use the
matrix forms but the dyadic ones above - we just use the dipole positions
as part of our simulation????).

The dual vectors are defined by $\mathrm{H^{ext\dagger}\tilde{H}^{ext}}=1$,
but inversion is not possible if J is not equal to N. We use the SVD
pseudo-inverse, defined by:

\[
\mathrm{H}=\mathrm{U\Sigma V^{\dagger}}\implies\mathrm{\tilde{H}}=\mathrm{U\Sigma^{-1}V^{\dagger}};\qquad\mathrm{H^{\dagger}\tilde{H}=U\Sigma\Sigma^{-1}U^{\dagger}}
\]
 which correctly accounts for numbers of source positions and sample
points. If the basis is complete, or overcomplete, the final relation
will equal the identity. If it is undercomplete, the modes are projected
onto a measurement space, a filter is applied (which may be impossible
to get rid of due to noise) and then the modes are reconstructed.

The whole process may be done using incremental SVD, where each new
measurement increases information until all degrees of freedom are
found.


\section{Modeling}


\subsection{Response Matrix}

All calculations are done in the basis with the fields or moments
at the dipoles unless otherwise stated. Firstly we calculate the local
succeptibility, called $\mathrm{X}$ rather than $\chi$ for clarity,
from \ref{eq:localx}. Next we apply the coupling and scattering from
...


\section{Results and Discussion}


\subsection{General Dipole Systems}

Ferromagnetic resonance first, then scan source frequency, then show
different source locations, then fringes, then coherence lengths/volumes


\subsection{EAI Reconstruction}


\subsection{Separate Modes}


\section{Conclusion}
\end{document}
