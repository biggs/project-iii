%% LyX 2.1.2.2 created this file.  For more info, see http://www.lyx.org/.
%% Do not edit unless you really know what you are doing.
\documentclass[english]{article}
\usepackage[latin9]{inputenc}
\usepackage{amsbsy}
\usepackage{amssymb}
\usepackage{babel}
\begin{document}
Typed up theory
\begin{enumerate}
\item \textbf{Non-interacting case}: For a collection of non-interacting
dipoles, it is found that:where
\end{enumerate}
\[
\overline{\overline{\chi}}(\boldsymbol{r})=\sum_{j}\frac{\gamma_{j}\mu_{0}m_{j}^{s}\omega}{(\omega_{0j}^{2}-\omega^{2})-i\omega\Gamma_{j}}\overline{\overline{S}}_{j}\delta(\boldsymbol{r-\boldsymbol{r_{0j}}})
\]


and 
\[
\overline{\overline{S}}_{j}=\frac{\omega_{0j}}{\omega}\boldsymbol{\hat{i}\hat{i}}-i\boldsymbol{\hat{j}\hat{i}}+i\boldsymbol{\hat{i}\hat{j}}+\frac{\omega_{0j}}{\omega}\boldsymbol{\hat{j}\hat{j}}
\]


is the spin dyadic.

When we expand in the basis of ${\textstyle h}{}^{e}\in\mathbb{C}^{2J}$
where ${\textstyle h}^{e}$ are the cartesian components of the external
field at the dipoles, $\chi$ becomes a block diagonal matrix. This
represents the system in the non-interacting case.

\textbf{2. Self interacting case}

The exchange interactions are added using another matrix, ${\textstyle G^{x}}$-
in the case of nearest neighbour interactions, simply the sum over
local dipole moments (represented in the vector $m\in\mathbb{C}^{2J}$
). Thus:

\[
{\textstyle m}=\chi{\textstyle h^{e}}+\chi{\textstyle G^{x}m}=(\chi^{-1}-{\textstyle G^{x}})^{-1}{\textstyle h^{e}}=\xi{\textstyle h^{e}}
\]


\textbf{3. Case with Scattering}

The scattered dipole field of a 
\end{document}
